\chapter{Electromagnetism}
\label{13}
\section{History}
As early as 600 B.C. the Greeks knew that a certain form of iron ore, now known as magnetite or lodestone, had the property of attracting small pieces of iron. Later, during the Middle Ages, crude navigational compasses were made by attaching pieces of lodestone to wooden splints. These splints always come to rest pointing in a N—S direction, and were the forerunners of the modern aircraft and ship compasses.

The word ‘lodestone’ is derived from an old English word meaning way, and refers to the directional property of the stone mentioned above.Chemically,it consists of iron oxide having the formula Fe\textsubscript{3}O\textsubscript{4}. The word magnetism is derived from Magnesia,the place where magnetic iron ore was first discovered.

In 1820, a Danish Physicist Hans Christian Oersted made one of the most important discovery of all times. He determined that when a current carrying wire is held near a compass needle, the needle is deflected. This discovery leads to the entire field of electromagnetism.

\section{Definition of Electromagnetism}
\textit{\textbf{“The branch of Physics which deals with the study of magnetic effects of electric current is called electromagnetism.”}}
\subsection*{Explanation}
The electric and magnetic fields are different aspects of electromagnetism but intrinsically related. Thus, a changing electric field generates a magnetic field and conversely  a changing magnetic field generates an electric field. The latter effect is called electromagnetic induction and is the basic operation for electric generators, induction motors and transformers and is studied in electromagnetism (we will cover it in next chapter).
\section{Magnetic Field}
\textit{\textbf{“The space around a magnet or current carrying conductor,where a test magnet can feel a force of attraction  or repulsion is called magnetic field.”}}
\subsection{Forces of Magnets}
Magnets exert forces on each other.These forces are either attraction or repulsion.The effects may be summarised in the law of magnets:

\textit{\textbf{“Like poles repel and unlike poles attract.”}}
\subsection{Magnetic Field Lines}
\textit{\textbf{“Magnetic field lines are the curves drawn so that the tangent to a given curve at a point gives the direction of magnetic field at that point.”}}
\subsubsection{Properties of Field Lines}
Magnetic field are not visible but they can be represented by lines of magnetic force extending in three dimensions.The properties of magnetic lines of force are given as:
\begin{enumerate}[label = (\roman*)]
    \item The magnetic field lines start at a north pole and end at a south pole.
    \item These lines are smooth curves,they never cross or touch.(Can you state why?)
    \item The strength of the field is indicated by the distance between the lines\_closer lines mean a stronger field and vice versa.
    \item Magnetic field lines always form closed curves.
\end{enumerate}

\subsection{Magnetic Field of an Electric current}
As it is known that all electric currents produce magnetic fields.The size and shape of magnetic field depends on the size of the current and the shape(configuration) of the conductor through which the current is travelling.

\subsubsection{Magnetic field of a Straight Current Carrying Wire}
The magnetic field due to a straight wire may be plotted using the apparatus shown in the figure below.Iron fillings are sprinkled on a horizontal board and current is passed through the wire as a result of which a magnetic field will be produced.Iron fillings will be in the indicated pattern showing the magnetic field around a straight current carrying wire will be in the form of concentric circles.
The separation of lines increases with the distance from the wire,indicating the field is decreasing in strength as we move away from the wire.The field also increases as the current is increased in the wire.The direction of field can be found by placing magnetic compasses or using right hand rule which states that:

\textit{\textbf{“Imagine hold the conductor in the right hand with the thumb pointing in the direction of the current,the curled fingers will point in the direction of field.”}}

\section{Force on a current carrying conductor}
The interaction of magnetic fields produced by two magnets causes force of attraction or repulsion between the two.If a conductor is placed between the poles and a current is passed through the conductor,the magnetic fields of the current-carrying conductor and the magnet may interact,causing forces between them.In order to explain,we will demonstrate it by the following experiment:

Place a straight wire between the poles of a magnet.
When a current flows in the wire,a force is exerted on the wire.
In first demonstration,the current flows inward direction (into the page),the wire experiences a downward push.This force is neither parallel to magnetic field nor parallel to the wire.Instead this force is directed at right angle to the magnetic field and wire.Now,if the current is reversed(out of the page),the direction of push will also be reversed i.e. upward.It is found that the direction of force is
always perpendicular to the wire and also perpendicular to the direction of field.





These demonstrations lead us to define a rule for the direction of force,i.e.

\textit{\textbf{“Outstrech the fingers of your right hand in the direction of current,then bend the fingers in the direction of magnetic field,the extend thumb will indicate the direction of the force in the current carrying wire”}}


\subsection{Magnitude of Force}
 It is found experimentally that the magnitude of the force is directly proportional to :
 
 \begin{enumerate}[label = (\roman*)]
    \item Current in the wire
    \item Length of the wire inside the magnetic field
    \item Strength of the field
\end{enumerate}
i.e.

\begin{equation}
 F \propto ILB \nonumber
\end{equation}

\textbf{Secondly,it was also found that :}
 \begin{itemize}
    \item When the wire was perpendicular to the field,the force was maximum
    \item When the wire as parallel to to the field,there was no force at all
    \item At any angle,it varies with the sine of the angle between $\vec{L}$ and $\vec{B}$.
\end{itemize}

So,
\begin{equation}
  F \propto sin\theta \nonumber
\end{equation}
\begin{equation}
  F \propto ILBsin\theta \nonumber
\end{equation}
Here constant of proportionality is 1,so
\begin{equation}
  F = ILBsin\theta \nonumber
\end{equation}
\subsubsection{Generalized Form}
The magnitude as well as the direction of the magnetic force on a current carrying wire can be described in vector notations by the following cross product:
\begin{equation}
  \vec{F} =I \vec{L}\times \vec{B} \nonumber
\end{equation}

\begin{equation}
\vec{F} =ILBsin\theta \hat{n} \nonumber
\end{equation}
Where,
$\vec{L}$ is a vector whose magnitude is the length of the wire
and whose direction is along the wire(assumed straight) in the
direction of current.The unit vector $\hat{n}$ is along the direction
of $\vec{F}$ and is perpendicular to $\vec{L}$, $\vec{B}$ and 
plane determined by $\vec{L}$ and $\vec{B}$.

\subsubsection{Fleming's Left-Hand Rule}
The rule we discussed earlier is known to us already if we have 
a knowledge  of cross product (extending fingers in the direction of 
first vector in the cross product and curl them towards other vector,
thumb will indicate the direction of the resultant of this cross
product i.e. the direction of $\hat{n}$. An alternate rule for the
direction of force is the Fleming’s left hand rule which states that :

\textit{\textbf{“If the forefinger,central finger and thumb of left hand are held mutually perpendicular with the Forefinger pointing in the direction of Field,Central finger in the direction of Current,the thumb would indicate the Motion of conductor(the direction of force on conductor).”}}

\subsubsection{Definition of $\vec{B}$}
$\vec{B}$ being a vector has magnitude as well as direction.

\subsubsection{Direction of $\vec{B}$}
The direction of $\vec{B}$ at any point of the magnetic field is the direction in which the force acting on a straight current carrying wire,placed at that point,is zero, i.e.  $\vec{L}$ || $\vec{B}$.

As we know that when the wire is in the direction of field,it experiences no force,so we move wire so that the point comes when it experiences no force,we say that the direction of the wire at that point will be the direction of field.

\subsubsection{Magnitude of $\vec{B}$}
The magnitude of $\vec{B}$ is defined when angle between $\vec{B}$ and $\vec{L}$ is 90° and force is maximum,so,

\begin{equation}
  B = F_{max}/IL \nonumber
\end{equation}
So, \textit{\textbf{“It is the maximum force acting on a conductor of unit length when one ampere current passes through it”}}

\subsection{Unit of Magnetic Field}
The SI unit of magnetic field  is tesla (T) and :
\begin{equation}
  1T = INA^{-1}m^{-1} \nonumber
\end{equation}

\subsubsection{One tesla}
\textit{\textbf{“Magnetic field at any point is said to be one tesla if it exerts a force of 1 N on one metre length of the conductor placed at right angles to the field when a current of 1 A passes through it.”}}
\subsubsection{Other units}
An older name of tesla is weber per metre squared i.e. Wb/$m^{2}$. Another commonly used unit is gauss (G).
And,


\section{Magnetic Flux}
\textit{\textbf{“The number of magnetic field lines passing through a surface
is known as magnetic flux”}}
\begin{center}
    OR
\end{center}
\textit{\textbf{“The dot product of magnetic induction $\vec{B}$ and vector
area element $\vec{A}$ is known as magnetic flux.”}}
\paragraph{Symbol:}
Its symbol is $\phi_{B}$.

\subsection{Mathematical Form}
If $\vec{B}$ is the magnetic induction and $\vec{A}$ is the area
vector (a vector having magnitude equal to the area of surface and
direction normal to the area element), then the flux would be:
\begin{equation}
  \phi_{B} = BA cos\theta \nonumber
\end{equation}
where $\theta$ is the smaller angle between $\vec{B}$ and $\vec{A}$.

\subsection{Explanation}
As magnetic flux for magnetic induction $\vec{B}$ and element of area $\vec{\Delta A}$ is given by:
\begin{equation}
  \phi_{B} = B\Delta A cos\theta \nonumber
\end{equation}

We know that area vector is normal to the plane of area.If the area is not
a flat surface i.e. the angle between area vector and magnetic induction is
different at different points, thus we divide area into smaller ‘n’ elements.
So, the total magnetic flux through the whole area placed in a field of magnetic
induction $\vec{B}$ is the sum of the contributions from the individual area
elements is given by:
\begin{equation}
 \phi_{T} =  \sum_{i=1}^{n} \Delta \phi_{i} \nonumber
\end{equation}
\begin{equation}
\phi_{T} =  \sum_{i=1}^{n} B_{i}\Delta A_{i}cos\theta_{i} \nonumber
\end{equation}
In a uniform field,
\begin{equation}
\phi_{T} = B \sum_{i=1}^{n} \Delta A_{i}cos\theta_{i} \nonumber  
\end{equation}

\subsection{Maximum Flux}
If the surface area is held normal to the field lines such that area vector
$\vec{A}$ is parallel to the field,then the maximum lines of force will pass
and flux will be maximum i.e.
\begin{equation}
\phi_{B} =  BA cos0^{\circ} \nonumber
\end{equation}
\begin{equation}
\phi_{B} = BA \nonumber  
\end{equation}

\subsection{Minimum Flux}
 If the surface area is placed such that it is parallel to the
 lines of force, so that area vector $\vec{A}$ is normal to the field,
 no lines will pass,θ will be 90 degrees,flux will be zero.
 
\begin{equation}
\phi_{B} =  BA cos90^{\circ} \nonumber
\end{equation}
\begin{equation}
\phi_{B} = 0 \nonumber  
\end{equation}

\subsection{Unit}
Unit of magnetic flux is T$m^{2}$ called as weber(Wb).

\subsection{Magnetic Flux Density}
Using equation
\begin{equation}
B= \frac{\phi}{A} \nonumber  
\end{equation}
So magnetic induction B can also be defined as:

\textit{\textbf{“Magnetic flux per unit area.”}}.
Hence it is also called magnetic flux density. It has unit Wb/$m^{2}$ (T).

\section{Ampere’s Circuital Law}
\subsection{Background}
 We know that a current carrying wire has a magnetic field around it.
 The direction of the field can be determined by right hand rule.
 The magnitude of the field can be determined by a relation called 
 Ampere’s circuital law.
 \subsection{Statement}
 \textit{\textbf{“The sum of the dot products ‘B’ and ‘L’ around a closed path in the magnetic field of a current is equal to μo times the current enclosed by the path.”}}
 
 \subsection{Mathematically}
 \begin{equation}
\sum \vec{B}\cdot\Delta\vec{L}= \mu_{0}I \nonumber  
\end{equation}

\subsection{Explanation}
Let us first consider a special case of the magnetic field of a
long straight-current carrying wire as shown:
%Figure
From experiments as well as from the cylindrical symmetry of the wire,
it is obvious that the magnitude of magnetic induction is constant on
a circle of radius ‘r’ centred on wire.It is further observed that ‘B’
around a long straight current-carrying wire is directly proportional
to the current ‘I’ and inversely proportional to the distance ‘r’ from
the wire i.e.
\begin{equation}
    B \propto I \nonumber
\end{equation}
\begin{equation}
    B \propto 1/r \nonumber
\end{equation}
\begin{equation}
    B \propto I/r \nonumber
\end{equation}
\begin{equation}
    B \propto \mu_{0}I/2\pi r \nonumber
\end{equation}

