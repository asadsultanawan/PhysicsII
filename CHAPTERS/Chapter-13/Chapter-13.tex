\chapter{Electromagnetism}
\label{13}
\section{History}
As early as 600 B.C. the Greeks knew that a certain form of iron ore, now known as magnetite or lodestone, had the property of attracting small pieces of iron. Later, during the Middle Ages, crude navigational compasses were made by attaching pieces of lodestone to wooden splints. These splints always come to rest pointing in a N—S direction, and were the forerunners of the modern aircraft and ship compasses.

The word ‘lodestone’ is derived from an old English word meaning way, and refers to the directional property of the stone mentioned above. Chemically, it consists of iron oxide having the formula $Fe\textsubscript{3}O\textsubscript{4}$. The word magnetism is derived from Magnesia, the place where magnetic iron ore was first discovered.

In 1820, a Danish Physicist Hans Christian Oersted made one of the most important discovery of all times. He determined that when a current carrying wire is held near a compass needle, the needle is deflected. This discovery leads to the entire field of electromagnetism.

\section{Definition of Electromagnetism}
\textit{\textbf{“The branch of Physics which deals with the study of magnetic effects of electric current is called electromagnetism.”}}
\subsection*{Explanation}
The electric and magnetic fields are different aspects of electromagnetism but intrinsically related. Thus, a changing electric field generates a magnetic field and conversely a changing magnetic field generates an electric field. The latter effect is called electromagnetic induction and is the basic operation for electric generators, induction motors and transformers and is studied in electromagnetism. (we will cover it in next chapter).
\section{Magnetic Field}
\textit{\textbf{“The space around a magnet or current carrying conductor, where a test magnet can feel a force of attraction  or repulsion is called magnetic field.”}}
\subsection{Forces of Magnets}
Magnets exert forces on each other. These forces are either attraction or repulsion. The effects may be summarised in the law of magnets:

\textit{\textbf{“Like poles repel and unlike poles attract.”}}
\subsection{Magnetic Field Lines}
\textit{\textbf{“Magnetic field lines are the curves drawn so that the tangent to a given curve at a point gives the direction of magnetic field at that point.”}}
\subsubsection{Properties of Field Lines}
Magnetic field are not visible but they can be represented by lines of magnetic force extending in three dimensions. The properties of magnetic lines of force are given as:
\begin{enumerate}[label = (\roman*)]
    \item The magnetic field lines start at a north pole and end at a south pole.
    \item These lines are smooth curves,they never cross or touch.(Can you state why?)
    \item The strength of the field is indicated by the distance between the lines\_closer lines mean a stronger field and vice versa.
    \item Magnetic field lines always form closed curves.
\end{enumerate}
%figure



\subsection{Magnetic Field of an Electric current}
As it is known that all electric currents produce magnetic fields. The size and shape of magnetic field depends on the size of the current and the shape(configuration) of the conductor through which the current is travelling.

\subsubsection{Magnetic field of a Straight Current Carrying Wire}
The magnetic field due to a straight wire may be plotted using the apparatus shown in the figure below. Iron fillings are sprinkled on a horizontal board and current is passed through the wire as a result of which a magnetic field will be produced.Iron fillings will be in the indicated pattern showing the magnetic field around a straight current carrying wire will be in the form of concentric circles.
The separation of lines increases with the distance from the wire,
indicating the field is decreasing in strength as we move away from the wire.The field also increases as the current is increased in the wire. The direction of field can be found by placing magnetic compasses or using right hand rule which states that:

\textit{\textbf{“Imagine hold the conductor in the right hand with the
thumb pointing in the direction of the current, the curled fingers will
point in the direction of field.”}}

\section{Force on a current carrying conductor}
The interaction of magnetic fields produced by two magnets causes force
of attraction or repulsion between the two. If a conductor is placed
between the poles and a current is passed through the conductor, the magnetic fields of the current-carrying conductor and the magnet may interact, causing forces between them. In order to explain, we will demonstrate it by the following experiment:

Place a straight wire between the poles of a magnet. When a current flows in the wire, a force is exerted on the wire. In first demonstration, the current flows inward direction (into the page), the wire experiences a downward push. This force is neither parallel to magnetic field nor parallel to the wire. Instead this force is directed at right angle to the magnetic field and wire. Now, if the current is reversed (out of the page), the direction of push will also be reversed i.e. upward. It is found that the direction of force is
always perpendicular to the wire and also perpendicular to the direction of
field.
%figure


These demonstrations lead us to define a rule for the direction of 
force, i.e.

\textit{\textbf{“Outstretch the fingers of your right hand in the direction
of current, then bend the fingers in the direction of magnetic field,
the extend thumb will indicate the direction of the force in the
current carrying wire”}}

\subsection*{Magnitude of Force}
 It is found experimentally that the magnitude of the force is directly
 proportional to :

\begin{enumerate}[label = (\roman*)]
    \item Current in the wire
    \item Length of the wire inside the magnetic field
    \item Strength of the field
\end{enumerate}
i.e.
\begin{equation}\label{rel:13.1}
 F \propto ILB 
\end{equation}
Secondly, it was also found that:
\begin{itemize}
\item When the wire was perpendicular to the field, the force was maximum
\item When the wire as parallel to to the field, there was no force at all
\item At any angle, it varies with the sine of the angle between $\vec{L}$
and $\vec{B}$.
\end{itemize}
\noindent So, we can write:
\begin{equation}\label{rel:13.2}
  F \propto sin\theta 
\end{equation}
Combining relations \ref{rel:13.1} and \ref{rel:13.2}:
\begin{equation}\nonumber
  F \propto ILB sin\theta 
\end{equation}
Here constant of proportionality is 1, so
\begin{mybox}{red}{}
\begin{equation}\label{eq:13.3}
  F = ILB sin\theta 
\end{equation}
\end{mybox}
\subsubsection{Generalized Form}
The magnitude as well as the direction of the magnetic force on a current carrying wire can be described in vector notations by the following cross product:
\begin{mybox}{red}{}
\begin{equation}\label{eq:13.4}
  \vec{F} = I \vec{L}\times \vec{B}
\end{equation}
\end{mybox}
\begin{mybox}{red}{}
\begin{equation}\label{eq:13.5}
\vec{F} = ILB sin\theta\:\hat{n} 
\end{equation}
\end{mybox}
Where $\vec{L}$ is a vector whose magnitude is the length of the wire
and whose direction is along the wire (assumed straight) in the
direction of current. The unit vector $\hat{n}$ is along the direction
of $\vec{F}$ and is perpendicular to $\vec{L}$, $\vec{B}$ and 
plane determined by $\vec{L}$ and $\vec{B}$.
\subsection*{Fleming's Left-Hand Rule}
The rule we discussed earlier is known to us already if we have 
a knowledge of cross product (extending fingers in the direction of 
first vector in the cross product and curl them towards other vector, thumb will indicate the direction of the resultant of this cross
product i.e. the direction of $\hat{n}$. An alternate rule for the
direction of force is the Fleming’s left hand rule which states that:
\newline
\textit{\textbf{“If the forefinger, central finger and thumb of left hand are held mutually perpendicular with the Forefinger pointing in the direction of Field, central finger in the direction of Current, the thumb would indicate the Motion of conductor(the direction of force on conductor).”}}

\subsection*{Definition of $\vec{B}$}
$\vec{B}$ being a vector has magnitude as well as direction.

\subsubsection{Direction of $\vec{B}$}
The direction of $\vec{B}$ at any point of the magnetic field is the direction in which the force acting on a straight current carrying wire, placed at that point, is zero, i.e. $\vec{L}\parallel\vec{B}$.

As we know that when the wire is in the direction of field, it experiences no force, so we move wire so that the point comes when it experiences no force, we say that the direction of the wire at that point will be the direction of field.

\subsubsection{Magnitude of $\vec{B}$}
The magnitude of $\vec{B}$ is defined when angle between $\vec{B}$ and $\vec{L}$ is $90^{\circ}$ and force is maximum, hence,
\begin{equation}
  B = \frac{F_{max}}{IL}
\end{equation}
So, \textit{\textbf{“It is the maximum force acting on a conductor of unit length when one ampere current passes through it”}}

\subsection*{Unit of Magnetic Field}
The SI unit of magnetic field  is tesla ($T$) and :
\begin{equation}
  1\:T = 1\:NA^{-1}m^{-1} 
\end{equation}

\subsubsection{One Tesla}
\textit{\textbf{“Magnetic field at any point is said to be one tesla if it exerts a force of 1 N on one metre length of the conductor placed at right angles to the field when a current of 1 A passes through it.”}}
\subsubsection{Other Units}
An older name of tesla is weber per metre squared i.e. $\frac{Wb}{m^{2}}$. Another commonly used unit is gauss (G).
And,
\begin{center}
    $1\:G = 10^{-4}\:T$
\end{center}
Magnetic field of earth is half a gauss:
\begin{center}
    $\frac{1}{2}G = 0.5 \times 10^{-4}\:T$
\end{center}
\section{Magnetic Flux}
\textit{\textbf{“The number of magnetic field lines passing through a surface
is known as magnetic flux”}}
\begin{center}
    OR
\end{center}
\textit{\textbf{“The dot product of magnetic induction $\vec{B}$ and vector
area element $\vec{A}$ is known as magnetic flux.”}}
\subsection{Symbol}
Its symbol is $\Phi_{B}$.

\subsection*{Mathematical Form}
If $\vec{B}$ is the magnetic induction and $\vec{A}$ is the area
vector (a vector having magnitude equal to the area of surface and
direction normal to the area element), then the flux would be:
\begin{mybox}{red}{}
\begin{equation}\label{eq:13.8}
  \Phi_{B} = BA cos\theta 
\end{equation}
\end{mybox}
where $\theta$ is the smaller angle between $\vec{B}$ and $\vec{A}$.
\subsection*{Explanation}
As magnetic flux for magnetic induction $\vec{B}$ and element of area $\vec{\Delta A}$ is given by:
\begin{equation}\nonumber
  \Phi_{B} = B\Delta A cos\theta 
\end{equation}

We know that area vector is normal to the plane of area. If the area is not
a flat surface i.e. the angle between area vector and magnetic induction is
different at different points, thus we divide area into smaller ‘$n$’ elements.
So, the total magnetic flux through the whole area placed in a field of magnetic induction $\vec{B}$ is the sum of the contributions from the individual area elements is given by:
\begin{equation}\nonumber
 \Phi_{T} =  \sum_{i=1}^{n} \Delta \Phi_{i} 
\end{equation}
\begin{mybox}{red}{}
\begin{equation}\label{eq:13.9}
\Phi_{T} =  \sum_{i=1}^{n} B_{i}\Delta A_{i}cos\theta_{i}
\end{equation}
\end{mybox}
\noindent In a uniform field,
\begin{mybox}{red}{}
\begin{equation}\label{eq:13.10}
\Phi_{T} = B \sum_{i=1}^{n} \Delta A_{i}cos\theta_{i}  
\end{equation}
\end{mybox}
\subsection*{Maximum Flux}
If the surface area is held normal to the field lines such that area vector
$\vec{A}$ is parallel to the field, then the maximum lines of force will pass
and flux will be maximum i.e.
\begin{equation}\nonumber
\Phi_{B} =  BA cos0^{\circ} 
\end{equation}
\begin{equation}\nonumber
\Phi_{B} = BA   
\end{equation}

\subsection*{Minimum Flux}
 If the surface area is placed such that it is parallel to the
 lines of force, so that area vector $\vec{A}$ is normal to the field,
 no lines will pass,$\theta$ will be 90 degrees, flux will be zero.
 
\begin{equation}\nonumber
\Phi_{B} =  BA cos90^{\circ} 
\end{equation}
\begin{equation}\nonumber
\Phi_{B} = 0 
\end{equation}

\subsection*{Unit}
Unit of magnetic flux is $Tm^{2}$ called as weber (Wb).

\subsection{Magnetic Flux Density}
Using equation
\begin{mybox}{red}{}
\begin{equation}\label{eq:13.11}
B= \frac{\Phi}{A}  
\end{equation}
\end{mybox}
So magnetic induction $B$ can also be defined as:
\textit{\textbf{“Magnetic flux per unit area.”}}
Hence it is also called magnetic flux density. It has unit $\frac{Wb}{m^{2}}$ (T).

\section{Ampere’s Circuital Law}
\subsection*{Background}
 We know that a current carrying wire has a magnetic field around it.
 The direction of the field can be determined by right hand rule.
 The magnitude of the field can be determined by a relation called 
 Ampere’s circuital law.
 \subsection*{Statement}
 \textit{\textbf{“The sum of the dot products ‘$B$’ and ‘$L$’ around a closed path in the magnetic field of a current is equal to $\mu_{o}$ times the current enclosed by the path.”}}
\subsection*{Mathematical Form}
\begin{mybox}{red}{}
\begin{equation}
\sum \vec{B}\cdot\Delta\vec{L}= \mu_{0}I 
\end{equation}
\end{mybox}
\subsection*{Explanation}
Let us first consider a special case of the magnetic field of a
long straight-current carrying wire as shown:
%Figure

From experiments as well as from the cylindrical symmetry of the wire, it is obvious that the magnitude of magnetic induction is constant on
a circle of radius ‘$r$’ centred on wire. It is further observed that ‘$B$’
around a long straight current-carrying wire is directly proportional
to the current ‘$I$’ and inversely proportional to the distance ‘$r$’ from
the wire i.e.
\begin{equation}\label{rel:13.3}
    B \propto I 
\end{equation}
\begin{equation}\label{rel:13.4}
    B \propto \frac{1}{r} 
\end{equation}
Combining relations \ref{rel:13.3} and \ref{rel:13.4}:
\begin{equation}\nonumber
    B \propto \frac{I}{r}
\end{equation}
Introducing constant of proportionality,
\begin{mybox}{red}{}
\begin{equation}\label{eq:13.5}
    B = \frac{\mu_{0}I}{2\pi r}
\end{equation}
\end{mybox}
Where, $\frac{\mu_{0}}{2\pi}$ is constant of proportionality and its value is $4\pi\times10^{-7}\:WbA^{-1}m^{-1}$ and is called its permeability of free space.
Equation   shows radial dependency of $B$. This radial dependence is used to derive expression for Ampere’s law.
\subsection*{Derivation}
Let us consider a circle of radius ‘$r$’ around current carrying wire as shown below:
%figure

To find $\vec{B} \cdot \vec{L}$, where $L$ is the circumference of circle, we divide this path into small segments $\vec{\Delta{L_{1}}}$, $\vec{\Delta{L_{2}}}$, $\vec{\Delta{L_{3}}}$, ..., $\vec{\Delta{L_{n}}}$. Then:
%figure

\begin{equation}\nonumber
\vec{B} \cdot \vec{L}  =  \sum_{i=1}^{n} \vec{B_{i}}\cdot\Delta\vec{L_{i}}
\end{equation}
As it is clear from the figure that $\vec{B}$ is parallel to the $\vec{\Delta L}$ at each point, therefore:
\begin{equation}\nonumber
\sum_{i=1}^{n}\vec{B_{i}}\cdot\Delta\vec{L_{i}}=\sum_{i=1}^{n}B_{i}\Delta{L_{i}}
\end{equation}
As B is constant, so we pull it out of summation:
\begin{equation}\nonumber
\sum_{i=1}^{n} \vec{B_{i}}\cdot\Delta\vec{L_{i}}=B\sum_{i=1}^{n}\Delta{L_{i}}
\end{equation}
As,
\begin{equation}\nonumber
    \sum_{i=1}^{n}\Delta{L_{i}}=2\pi r 
\end{equation}
and $B = r$ (for circle) and  $B = \frac{\mu_{0}I}{2\pi r}$
\begin{equation}
\begin{gathered}
\nonumber
\sum_{i=1}^{n}\vec{B_{i}}\cdot\Delta\vec{L_{i}}=\frac{\mu_{0}I}{2\pi r}\times 2\pi r
\end{gathered}
\end{equation}
\begin{mybox}{red}{}
\begin{equation}\label{eq:13.16}
    \sum_{i=1}^{n} \vec{B_{i}}\cdot\Delta\vec{L_{i}}= \mu_{0}I
\end{equation}
\end{mybox}
where,
‘$I$’ is the current closed.From equation, it is clear that $\sum\vec{B}\cdot\Delta\vec{L}$ is independent of the shape or size of the closed path. It can be applied to closed path of any shape.
\begin{mybox}{green}{}
\subsection*{\note{}Note:}
\begin{enumerate}[label = (\roman*)]
\item Ampere’s law can be applied to any assembly of currents. The closed path in the magnetic field is called amperian loop.
\item If there is no current enclosed within the amperian path, the amperian summation of $\sum\vec{B}\cdot\Delta\vec{L}$ is zero.
\end{enumerate}
\end{mybox}
\section{Magnetic Field due to a Current Carrying Solenoid}
\subsection*{Solenoid}
 \textit{\textbf{“If a straight wire is wrapped in the form of several closely spaced loops, the resulting device is called solenoid.”}}
 
 \section{Magnetic Force on a Moving Charge}
 We know that force acting on a conductor of length ‘$L$’ having current ‘$I$’ placed at right angles to the magnetic field ‘$B$’ is given by:
\begin{equation}\label{eq:13.17}
     F_{max} = ILB
\end{equation}
It is also our well-established knowledge that the conductor contains charges and current in a conductor is due to drift of the charges. Therefore, we conclude that the force on the conductor is due to the force on the charges in motion in the conductor.
If the charge on a particle is ‘$+q$’ and there are ‘$n$’ charges per unit length of the wire moving with velocity ‘$V$', then:
\newline
Distance moved by charges in one second will be:
\begin{center}
$S= V(1) = V\:m$
\end{center}
The charges contained in ‘$V$’ metres will pass through the section $PP^{\prime}$ in one second which makes the current ‘$I$’. The charge contained in $V$ m will be $nqV$, so current will be:
 \begin{equation}\label{eq:13.18}
    I = nqV
\end{equation}
Put value of `$I$' in equation \ref{eq:13.17}, we get:
\begin{equation}\nonumber
    F_{max} = nqLVB
\end{equation}
 The number of charges in unit length will be `$n$' and in length $L$, they will be `$nL$'. So, force on ‘$nL$’ charges is:
\begin{equation}\nonumber
     F_{max} = nqLVB
\end{equation}
 The force on each charge will be:
\begin{equation}\nonumber
     F_{max}=\frac{nqLVB}{nL}
\end{equation}
\begin{mybox}{red}{}
\begin{equation}\label{eq:13.19}
     F_{max} = qVB
\end{equation}
\end{mybox}
This was the special case when the conductor was placed perpendicular to the field,in our case, ‘$\vec{B}$’ is perpendicular to ‘$\vec{V}$’, but if there is an angle `$\theta$' between ‘$\vec{B}$’ and ‘$\vec{V}$’, then:
\begin{mybox}{red}{}
 \begin{equation}\label{eq:13.20}
     F = qVBsin\theta
\end{equation}
\end{mybox}
The equation gives the magnitude of the magnetic force on a particle of charge ‘$q$’ moving with velocity ‘$V$’ in a magnetic field of strength ‘$B$’ and `$\theta$' is the angle between ‘$\vec{B}$’ and ‘$\vec{V}$’.
\subsection*{Maximum Force}
When $\theta = 90^\circ,\:sin90^{\circ} = 1$, then:
\begin{equation}\nonumber
     F_{max} = qVB
\end{equation}
This means that the force is maximum when the charge particle moves perpendicular to the magnetic field.
\subsection*{Minimum Force}
When $\theta = 0^{\circ},\:sin0^{\circ}=0$, then:
 \begin{equation}\nonumber
     F = qVBsin0^{\circ}
\end{equation}
\begin{equation}\nonumber
     F=0
\end{equation}
This means that when the particle moves parallel to the field direction, the force will be minimum i.e zero.
\subsection*{Direction}
We know that force is a vector quantity, so it must have a specific direction. Equation \ref{eq:13.20} can be written in vector form as:
\begin{mybox}{red}{}
\begin{equation}
    \vec{F}=q(\vec{V} \times \vec{B})
\end{equation}
\end{mybox}
\begin{mybox}{red}{}
\begin{equation}
    \vec{F}=qVBsin\theta \hat{n}
\end{equation}
\end{mybox}
The direction of the force is perpendicular to velocity and magnetic field and the plane formed by them and can be determined by right hand rule of cross product. Interestingly, you can extend Fleming’s left hand rule to determine the direction of this force,
\textit{\textbf{``Hold your forefinger, middle finger and thumb of your left hand mutually perpendicular such that forefinger points in the direction of field, middle finger in the direction of velocity of a charged particle, then the thumb would indicate the direction of force on that moving charge."}}
\begin{mybox}{green}{}
\subsection*{\note{}Note:}
\begin{enumerate}[label = (\roman*)]
\item If the force on ‘$+q$’ is upward, then force on ‘$-q$’ will be downward in the same field.
\begin{equation}\nonumber
    \vec{F}=-qVBsin\theta \hat{n}
\end{equation}
\begin{equation}\nonumber
    \vec{F}=qVBsin\theta (-\hat{n})
\end{equation}
This means that the force on negative charge is opposite to that of positive charge.
\item As the force is perpendicular to the direction of motion,therefore it can only change the direction of motion of the charged particle. It can neither speeds up or slowed down the particle.
\end{enumerate}
\end{mybox}
\subsection*{Applications}
The deflection of charged particles by magnetic field is used in T.V tubes, electron microscopes, spectrographs and charged particles accelerators like Cyclotron and Betatron.

\subsection{Circular Trajectory of a Charged Particle in a Magnetic Field}
Let us consider a particle of charge ‘$+q$’ thrown perpendicular to a uniform magnetic field of magnetic flux density ‘$B$’ with velocity ‘$v$’ as shown:
%figure



The magnetic force in the above case:
\begin{equation}\label{eq:13.23}
    F_{m} = qvB
\end{equation}
As this force is all the time perpendicular to the direction of velocity and magnetic field, therefore it compels the charge to move in circular path. So, magnetic force is the necessary centripetal force i.e.
\begin{equation}\nonumber
    F_{m}=F_{c}
\end{equation}
Hence, from ]ref{eq:13.23}:
\begin{equation}\nonumber
    F_{m}=\frac{mv^{2}}{r}
\end{equation}
Putting values, we get: 
\begin{equation}\nonumber
    \frac{mv^{2}}{r} = qVB
\end{equation}
\begin{mybox}{red}{}
\begin{equation}\label{eq:13.24}
    r=\frac{mv}{qB}
\end{equation}
\end{mybox}
\noindent This equation gives the radius of the circular path in which a charge ‘$q$’ of mass ‘$m$’ and velocity ‘$V$’ will move in a magnetic field when projected perpendicularly.
Also,
\begin{equation}\nonumber
     v = r\omega
\end{equation}
Put this value of `$r$' from equation \ref{eq:13.24}, we get:
\begin{equation}\nonumber
    v = \frac{mr\omega}{qB}
\end{equation}
which gives:
\begin{mybox}{red}{}
\begin{equation}\label{eq:13.25}
    \omega=\frac{qB}{m}
\end{equation}
\end{mybox}
\noindent This is the angular frequency of the circulating body.
As,
\begin{equation}\nonumber
    \omega = 2 \pi f
\end{equation}

\begin{equation}\nonumber
    f=\frac{\omega}{2\pi}
\end{equation}
\begin{mybox}{red}{}
\begin{equation}\label{eq:13.26}
    f=\frac{qB}{2 \pi m}
\end{equation}
\end{mybox}
\noindent And,
\begin{equation}\nonumber
    f=\frac{1}{T}
\end{equation}
So,
\begin{mybox}{red}{}
\begin{equation}
    T=\frac{2\pi m}{qB}
\end{equation}
\end{mybox}

From the equation, it is clear that the time period of the particle is independent of the radius of the path followed by the charge. Smaller the radius, less will be the velocity and larger radius will increase the velocity as a result time period for a given charge will remain same.
The frequency found in equation \ref{eq:13.26} is called ‘cyclotron frequency’ (cyclotron is a device used for accelerating charged particles) of the circulating particle.
\subsection{Helical Trajectory of a Charged Particle}
If the direction of velocity is not perpendicular to the magnetic field, then instead of circular trajectory, charged particle adopts a helical path as shown:
%figure

Actually, velocity has two components, the vertical component and horizontal component is affected by magnetic force describing circular path and horizontal  moves it straight. As a result, it follows a spiral path.
\subsection{Determination of $e/m$ for an Electron}
\subsubsection{Principle}
Circular trajectory of charged particle in magnetic field.
\subsubsection{Mathematical Derivation}
A narrow beam of electrons moving with constant velocity ‘$v$’ is projected at right angles to a uniform magnetic field ‘$B$’ as shown:
%figure

The magnetic force provides the necessary centripetal force, So,
\begin{equation}\nonumber
    \frac{mv^{2}}{r}=eVB
\end{equation}
\begin{equation}\label{eq:13.28}
    \frac{e}{m}=\frac{v}{Br}
\end{equation}
Knowing $v$, $B$ and $r$, value of $\frac{e}{m}$ can be calculated.
\subsubsection{Determination of radius}
The radius is measured by making the electrons trajectory visible. This is done by filling a glass tube with a gas such as hydrogen at low pressure. The tube is placed in a region occupied by a uniform magnetic field of known value. As the electrons are shot into this tube, they begin to move along a circle under the action of magnetic force. As the electrons move, they collide with the atoms due to which they emit light and their path becomes visible as a circular ring. The diameter of this ring can be easily determined.
\subsubsection{Determination of Velocity}
We know that the kinetic energy gained by electrons is due to the electric potential.
\begin{equation}\nonumber
    Gain\:in\:Kinetic\:Energy= \frac{1}{2} mv^{2} = eV
\end{equation}
So,
\begin{equation}\nonumber
 v=\sqrt{\frac{2eV}{m}}   
\end{equation}

Now, to find $\frac{e}{m}$, squaring above equation:
\begin{equation}\nonumber
 v^{2}=\frac{2eV}{m}   
\end{equation}
Squaring equation \ref{eq:13.28}:
\begin{equation}\label{eq:13.29}
\frac{e^{2}}{m^{2}}=\frac{v^{2}}{B^{2}r^{2}}
\end{equation}
Putting value of $v^{2}$ in equation \ref{eq:13.29}:
\begin{equation}\nonumber
\frac{e^{2}}{m^{2}}=\frac{2e V}{m B^{2}r^{2}}
\end{equation}
\begin{mybox}{red}{}
\begin{equation}\label{eq.13.30}
\frac{e}{m}=\frac{2V}{B^{2}r^{2}}
\end{equation}
\end{mybox}
Using potential which is known and substituting all values ‘$\frac{e}{m}$’ can be calculated. The accurately known value of ‘$\frac{e}{m}$’ for electron is 1.77588 $\times$ $10^{11}$ $\frac{C}{kg}$.

\subsection{Charge in Combined Electric and Magnetic Field}
\subsubsection{Electric Force}
We know that force on a charge ‘$q$’ placed in an electric field of intensity ‘$E$’ is given by:
\begin{equation}\label{eq:13.31}
    F=q E
\end{equation}
Also, from Newton’s 2nd law:
\begin{equation}\label{eq:13.32}
    F=ma
\end{equation}
If charges are free to move, then acceleration will be:
\begin{equation}\nonumber
    ma=q E
\end{equation}\label{eq:13.33}
\begin{equation}
    a=\frac{q E}{m}
\end{equation}
If the electric field is uniform,the force is constant. The acceleration produced will be uniform. Therefore, the position and velocity of the particle at any instant of time can be determined by using the equations for uniformly accelerated motion. The figure shows how a beam of electrons is deflected by the uniform electric field.
%figure

\begin{mybox}{blue}{}
\subsection*{\checkpoint{} Checkpoint 13.1}
Prove that electrons would follow a parabolic path when deflected by electric field.
\end{mybox}
\subsubsection{Magnetic Force}
The force on a charge ‘$q$’ moving with velocity ‘$v$’ in a region of magnetic field ‘$B$’ is:
\begin{equation}\nonumber
    \vec{F}=q(\vec{v} \times \vec{B})
\end{equation}
As this force is always perpendicular to the motion of charged particle. So it will move the particle in circular path as shown:
%figure

\subsubsection{Combined Electric and Magnetic Field}
If a charge particle ‘$q$’ is projected in a region with velocity ‘$v$’ where there is magnetic field ‘$B$’ and electric field ‘$E$’, then force on the particle will be:
\begin{mybox}{red}{}
\begin{equation}\label{eq:13.34}
    \vec{F}=q\vec{E} + q(\vec{v} \times \vec{B})
\end{equation}
\end{mybox}
\noindent This force is called Lorentz force.
\begin{mybox}{green}{}
\subsubsection*{\note{}Note:}
The electric field can:
\begin{enumerate}
[label = (\alph*)]
\item Deflect the charge particle
\item Speed up or slow down the particle or impart energy to it
\end{enumerate}
whereas magnetic field can:
\begin{enumerate}[label = (\alph*)]
\item Deflect the charge particle
\item Not speed up or slow down the particle means can not change kinetic energy of the moving charge. This is because magnetic force is perpendicular to $v$.
\end{enumerate}
\end{mybox}
\subsection{Velocity Selector}
\textit{\textbf{“When the electric field is perpendicular to the magnetic field such that a charge particle with a particular velocity passes undeflected, the arrangement is called velocity selector.”}}

\subsubsection{Construction}
A velocity selector consists of a tube in which electric field ‘$E$’ is oriented perpendicular to the magnetic field ‘$B$’ as shown:
%figure

The field strengths ‘$E$’ and ‘$B$’ are so oriented that the electric force and the magnetic force act in opposite direction. A charged particle that enters the tube in a direction perpendicular to both ‘$E$’ and ‘$B$’ with speed $v=\frac{E}{B}$ will pass undeflected.

\subsubsection{Derivation}
As discussed above, the electric and magnetic forces are equal and opposite for velocity selector, so,
\begin{equation}\nonumber
    F_{E}=F_{m}
\end{equation}
\begin{equation}\nonumber
    qE = qvB
\end{equation}
\begin{mybox}{red}{}
\begin{equation}\label{eq:13.35}
    v = \frac{E}{B}
\end{equation}
\end{mybox}
So charge particles with this velocity will pass undeflected and those with velocities other than this will be deflected either upwards or downwards.
