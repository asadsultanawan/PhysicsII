\chapter{Electrodynamics/Current Electricity}
\section{Electric Current}
\textit{\textbf{“The flow of charge per unit time is called electric current”.}}
\begin{center}
    \textbf{OR}
\end{center}
\textit{\textbf{“Whenever electric charge flows, current is said to exist.”}}
\subsection*{Symbol}
It is denoted by `I'.
\subsection*{Mathematical Form}
If `Q' is the amount of charge flowing through a wire of cross section ‘A’ in time ‘t’, then current ‘I’ will be given by:
\begin{equation}\label{eq:12.1}
    I = \frac{Q}{t} 
\end{equation}
\subsection*{Explanation}
We know that substances which conduct electricity are known as conductors and which do not, called as insulators. Before the discovery of electron, proton etc., scientist supposed that the current is due to the flow of positive charges. The concept of flow of positive charges was developed because in physics, the positive terminal is at high potential with respect to a negative terminal. With the discovery of electron and nucleus, it is clear that:
\begin{enumerate}[label = (\roman*)]
    \item In metal conductors, the current is due to the flow of electrons only.
    \item In liquids, current is due to flow of negative and positive charges, e.g in case of electrolytes.
    \item In discharge through gaseous, the current is due to the flow of both charges.
\end{enumerate}
When we talk about electric current, we often study the behaviour of metallic conductors. In metallic conductors, actually the electrons flow from negative terminal to positive terminal. However,prior to electron theory, it was assumed that positive charge flows only. But it was observed that a negative charge flowing in one direction has an equivalent effects if the same positive charge flows in opposite direction. So, the concept of flow of positive charge was retained as a convention and so called conventional current. The current due to electrons was called electronic current. The direction of electronic current is from negative terminal of the battery while that of conventional current is taken from positive to negative terminal of the battery.
%Figure
From now, we will take the direction of current as conventional direction 
i.e. from positive end to negative end.
\section{Drift Velocity of Electrons in a Metallic Conductor}
Every metal has a huge number of free electrons which wander randomly within the body of the conductor. The average speed of free electrons is sufficiently high, approximately of the order of 10\textsuperscript{5} m/s. During random motion,
the free electrons collide with the atoms of conductor again and again and after each collision,  their direction of motion changes. As many electrons move in one direction, as much electrons move in opposite direction. So due to random motion, there is no net flow of charges  in a particular direction, so no current exists in the case when no power supply is connected.
\subsection*{Motion of Charges When the Potential Difference is Applied}
When a potential difference is applied across the ends of a conductor, it sets up an electric field at every point along the wire. As inside the conductor, there are free charges (free electrons in case of metallic conductors), so electrons experience a force in a direction opposite to the electric field given by:
\begin{equation}\label{eq:12.2}
    \vec{F} = -e\vec{E}
\end{equation}
Due to this force, electrons gain velocity and hence accelerate, then Newton’s second law gives:
\begin{equation}\label{eq:12.3}
    \vec{F} = m\vec{a}
\end{equation}
This net force is actually the force exerted on electrons by the electric field, hence comparing equations \ref{eq:12.2} \& \ref{eq:12.3}, we get:
\begin{equation}
    m\vec{a} = -e\vec{E} \nonumber
\end{equation}
\begin{equation}\label{eq:12.4}
    \vec{a} = -\frac{e\vec{E}}{m}
\end{equation}
%Figure
So,direction of acceleration is in a direction opposite to electric field.As electrons move along the wire, they continuously collide with atoms, and due to the force by electric field, they acquire a net velocity along a direction, this velocity is called drift velocity and is defined as, \textit{\textbf{``the average velocity with which free electrons get drifted in a metallic conductor under the influence of electric field is called drift velocity."}} Symbolically, the drift velocity is denoted by $\vec{V_{d}}$.
\subsection*{Order of Drift Velocity}
The drift velocity of electrons in a metallic conductor is of the order of 10\textsuperscript{-5} m/s.
\subsection*{Effect Transfer}
As drift velocity of electrons is of the order of 10\textsuperscript{-5} m/s but when we switch on the bulb,it immediately glows. Although the drift velocity is negligible but the effect of electrons movement travel around the circuit with a speed approximately equal to the speed of light that is of the order of 10\textsuperscript{8} m/s. That’s why, when we switch on the bulb, the bulb glows in fractions of a second.
\section{Sources of Current}
\textit{\textbf{``A device that supply a constant current by maintaining a constant potential difference between its two terminals is called a source of current.”}}
\subsection*{Explanation}
It is a firmly established convention that a positively charged body is at higher potential than a negatively charged body. When a body at a higher potential is connected to a body at a lower potential through a metallic wire, electric current will flow from higher potential to lower potential as shown in the figure below:

%Figure

The current will stop when both the bodies come at the same potential. To maintain a steady current through the wire,the ends of the wire must be maintained at a constant potential difference. So, a device or agent must be present that will maintain the required potential difference. This device will convert some other forms of energy into electrical energy. Such a device is called a ‘power supply’ (or source of emf, we will talk about emf in this chapter). Examples are:
\begin{enumerate}[label=(\roman*)] 
    \item In electrical cells,chemical energy is converted into electrical energy.
    Similar is the principle of battery.
    \item Electric generator convert mechanical energy
    into electrical energy (we will talk about it in chapter 14).
    \item Thermocouple convert heat energy into light energy
    (we will discuss thermocouple in this chapter).
    \item Solar cells convert light energy int electrical
    energy (we will encounter solar cell in chapter 17).
\end{enumerate}

\section{Elecroencephalogram(EEG)}
%Write this section later
\section{Ohm's Law}
\subsection*{Background}
In previous section, we studied that when a potential difference is maintained across the ends of a wire, current flows. The relationship between potential difference(V) and electric current (I) in a D.C circuit was first discovered by German scientist George Simon Ohm in 1826 in the form of a law known as ‘Ohm’s law’. This law is known to be the fundamental law of electricity.






